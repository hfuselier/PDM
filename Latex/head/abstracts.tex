%\begingroup
%\let\cleardoublepage\clearpage


% English abstract
\cleardoublepage
\chapter*{Abstract}
%\markboth{Abstract}{Abstract}
\addcontentsline{toc}{chapter}{Abstract} % adds an entry to the table of contents
% put your text here
The failure mode and strength of rock are often evaluated using failure criteria that (i) neglect the intermediate principal stress or (ii) examine conditions over a limited range of mean stress. Review of the literature, however, shows that tests involving multi-axial stress states suggest that all three principal stresses should be considered in evaluating yield or failure. Further, rock displays a pressure dependence that can be interpreted as a change in friction with confinement. To explore the nature of stress states at failure, experiments and published data were analyzed for Dunnville sandstone within the framework of stress invariants $p$, $q$, and $\theta$, where $p$ = mean stress, $q$ = deviatoric stress, and $\theta$ = Lode angle. A series of conventional triaxial compression, conventional triaxial extension, and true-triaxial compression tests were conducted. Published data for Dunnville sandstone were collected and a database consisting of similar tests on Dunnville sandstone was developed. The axisymmetric compression data were fitted to three failure criteria: Mohr-Coulomb (MC), Hoek-Brown (HB), Paul-Mohr-Coulomb (PMC), a generalized linear criterion containing all three principal stresses. This criterion was evaluated using three parameter and six parameter formulations. The results show that the six-parameter PMC provides the best approximation of the test data and successfully captures, in a piecewise-linear manner, the well-known nonlinear nature of the failure surface. The Paul-Mohr-Coulomb criterion representation in the $(p-q)$ diagram involves a failure surface in principal stress space that can be described as a 6-12-6 sided pyramid. The thesis presents analyses, including data from multi-axial stress states, and discussion on the three failure criteria, including details of the generalized linear condition. 


% % German abstract
% \begin{otherlanguage}{german}
% \cleardoublepage
% \chapter*{Zusammenfassung}
% %\markboth{Zusammenfassung}{Zusammenfassung}
% % put your text here
% \lipsum[1-2]
% \vskip0.5cm
% Stichwörter: 
% %put your text here
% \end{otherlanguage}




% % French abstract
% \begin{otherlanguage}{french}
% \cleardoublepage
% \chapter*{Résumé}
% %\markboth{Résumé}{Résumé}
% % put your text here
% \lipsum[1-2]
% \vskip0.5cm
% Mots clefs: 
% %put your text here
% \end{otherlanguage}


%\endgroup			
%\vfill
