\chapter{Conclusion}\label{ch6:chapter6}
\section{Conclusion}

The purpose of this study was to evaluate the Mohr-Coulomb, Hoek-Brown and Paul-Mohr-Coulomb failure criteria by analyzing the results of multi-axial experiments performed on Dunnville sandstone. 

A review of the three failure criteria was provided, and the three were described in different coordinates systems: the $(p-q)$ , $(\sigma_3-\sigma_1)$ and $\pi$- planes. The strength parameters used are the frictions angles in compression $\phi_c$ and in extension $\phi_e$ , the theoretical isotropic tensile strength $V_0$, the uniaxial compressive strength $C_0$ and the fitting parameter $m$. The parameters for the Mohr-Coulomb and Hoek-Brown failure criteria were based on the results of axisymmetric triaxial compression experiments, and these two  do not include the intermediate stress $\sigma_{II}$ in their formulation. The three parameter Paul-Mohr-Coulomb failure criterion is defined by the three principal stresses, and its parameters were computed with the results of all multi-axial data, using a fitting method based on the least squares approach.

Dunnville sandstone was used as a representative rock with known geological history and mineralogy, and its behavior under uniaxial and multi-axial testing was observed by performing appropriate experiments. To investigate the intermediate stress effect on rock failure, two true-triaxial experiments under different stress paths were performed using the University of Minnesota Plane-Strain Apparatus. The addition of pistons enabling the direct application of the intermediate stress ($\sigma_{II}$) gave the possibility to perform a true-triaxial experiment with a plane strain testing condition ($\epsilon_2=0$). 

The results of the experiments performed in the scope of this study enriched the available data on Dunnville sandstone, and were gathered with results published in the literature. This database was then used to evaluate the Mohr-Coulomb, Hoek-Brown and three parameter Paul-Mohr-Coulomb failure criteria. Their comparison, based on the least mean standard deviation misfit, showed that the three parameter Paul-Mohr-Coulomb provided the best prediction and fitting of the experimental data, due to the inclusion of the intermediate stress and the definition of different frictions angles for compression and extension. 

The Paul-Mohr-Coulomb failure criterion was further investigated by evaluating its accuracy using a failure surface described by two planes instead of one, leading to a six parameter Paul-Mohr-Coulomb failure criterion. The failure criterion fitting was enabled by the development of a algorithm, that automatically allocated the data to the failure surface (either plane 1 or 2) that minimizes the mean squared error of the distance between the two-plane failure surface and the data points. This program was generalized to enable its use for any rock with available results of multi-axial strength experiments. 

The evaluation showed that Paul-Mohr-Coulomb with six parameters (two planes) provides the most accurate prediction and fitting of the failure surface.

\section{Future work}

As the evaluation of the failure criteria is based on the multi-axial experiments, the improvement of database is crucial. It is recommended that multi-axial experiments continue, where all three principal stresses at failure are known. Interesting stress states that need further investigation include true-triaxial experiments under (i) plane strain condition or (ii) constant mean stress and constant Lode angle. 