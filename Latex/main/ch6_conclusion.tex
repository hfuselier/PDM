\chapter{Conclusion}\label{ch6:chapter6}
\section{Conclusion}

The purpose of this study was to evaluate the Mohr-Coulomb, Hoek-Brown and Paul-Mohr-Coulomb failure criteria by analyzing the results of multi-axial experiments performed on Dunnville sandstone. 

First, a review the three failure criteria was provided, where they were presented thought the definition of their material parameters in different coordinates systems. The strength parameters used the criteria formulations are the frictions angles in compression $\phi_c$ and in extension $\phi_e$ , the theoretical isotropic tensile strength $V_0$, the uniaxial compressive strength $C_0$ and the fitting parameter $m$. The parameters for the Mohr-Coulomb and Hoek-Brown failure criteria computation was based on the results of axisymmetric triaxial experiments, as they do not include the intermediate stress $\sigma_{II}$ in their definition. The three parameter Paul-Mohr-Coulomb failure criteria is defined by the three principal stresses, and its parameters were computed with the results of all multi-axial data, using a fitting method based on the least squares solution.  Other parameters were defined to represent the failure criteria in the $(p-q)$ , $(\sigma_3-\sigma_1)$ and $\pi$- planes.

Dunnville sandstone was presented though its geological history and mineralogy, and its behavior under uniaxial and multi-axial testing was observed by performing a uniaxial compression test and several conventional (axisymmetric) triaxial experiments. In order to investigate the intermediate stress effect on rock failure, four true-triaxial experiments under different stress paths were performed using the University of Minnesota Plane-Strain Apparatus. The addition of pistons enabling the direct application of the intermediate stress ($\sigma_{II}$) gave the possibility to perform a true-triaxial experiment with a plane strain testing condition ($\epsilon_2=0$). 

The results of the experiments performed in the scope of this study enriched the available data from multi-axial experiments on Dunnville sandstone, and were gathered with results published in literature in a database. This database was then used to evaluate the Mohr-Coulomb, Hoek-Brown and three parameter Paul-Mohr-Coulomb failure criteria. Their comparison, based on the least mean standard deviation misfits, showed that the three parameter Paul-Mohr-Coulomb provided the best prediction and fitting of the experimental data, thanks to the inclusion of the intermediate stress and the definition of different frictions angles for compression and extension. 

The Paul-Mohr-Coulomb failure criterion was further investigated by evaluating its accuracy using a failure surface described by two planes instead of one, leading to a six parameter Paul-Mohr-Coulomb failure criterion. The failure criterion fitting was enabled by the development of a program, in the coding environment Python, that automatically allocated the data to the failure surface plane that minimize the mean squared error of the distance between the two-plane failure surface and the data points. This program was generalized to enable its use for all rocks with available results of multi-axial experiments. 

The evaluation of the three failure criteria for Dunnville sandstone showed that Paul-Mohr-Coulomb provide the most accurate prediction and fitting of the failure surface, whether it is made of one plane or two planes.

\section{Future work}

As the evaluation of the failure criteria is based on the multi-axial experiments results, the improvement and extension of rock databases is crucial. Future work on this subject could be to continue the investigation of the intermediate stress effect, by performing other true-triaxial experiments with plane strain condition and to diversify the testing conditions. An interesting modification of the experiments presented in this thesis could be to perform true-triaxial tests with a constant Lode angle condition instead of a mean stress one. This could be a way to investigate the effect of the Lode angle value on failure of rocks. 