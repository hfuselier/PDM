\chapter{Introduction}
\section{Introduction}

The wide variety of rocks that forms the earth crust makes it challenging to model. Each formation that exist beneath us can in itself be considered as different materials, with diverse mineralogy, geological histories and behavior. However, their one common thread is a composite structure formed by minerals, pores and cracks [1]. Years of worldwide experimentations on these materials already showed similarities in results, through the definition of material parameters used to characterize their behavior. Analysis and interpretation of these experimental outcomes are key elements in the understanding of rock response, and enables safer prediction of their behavior, which is used in the design of geotechnical structures. 

Rock response under a certain state of stress is associated with material strength and is often characterized by its failure. Multiple experiments, such as conventional triaxial and multiaxial tests, were developed with the aim of getting more information on the various states of stress that leads to rock failure. However, site-specific rock engineering properties are often not available during the preliminary design phase of structures constructed on or into the rock mass. Therefore, predictive models are widely used in initial stage of the design for prediction of the engineering properties of rock mass which may have an empirical or theoretical nature. 

Over the last century, many predictive models have been developed. They describe planes in the $\sigma_1$-$\sigma_2$-$\sigma_3$ space that approximate experiments data using material parameters. The plethora of failure criterion theories respond to the challenge of finding one that can give the most accurate description of rock behavior. Empirical models are usually developed for a specific rock type or rock formation and therefore, need to be evaluated before they can be applied to estimate the properties of rock masses that were not included in the data used for their development. Theoretical models have been developed to rigorously estimate the stress state at which rocks experience failure \cite{Labuz2018}. Such models are often developed based on fundamental aspects of rock mechanics and there are applicable to a wider range of rocks and rock formations.

The most well-known and widely used failure criterion is the Mohr-Coulomb model, which provides a linear relationship between the normal stress and stress strength of materials. The failure envelop is characterized using the friction angle $\phi$ and the cohesion intercept $c$ \cite{Jaeger1979}. Other criterion, such as the Hoek-Brown model for intact rocks and rock masses, are empirical and non-linear [3]. They give relatively reasonable approximation of the state of stress at failure, especially for low values confining stress (i.e., minor principal stress, $\sigma_3$). These failure criteria may be written in terms of the major ($\sigma_1$) and minor ($\sigma_3$) principal stresses, without any consideration to effect of intermediate principal stress ($\sigma_2$). 

\section{Limitation of current failure criteria and motivation}


Experiments, however, have shown that the intermediate principal stress affects the mode of failure and the major principal stresses that are developed when the rock mass fails \cite{Labuz2018,Labuz1996, Zeng2019,Makhnenko2013}. Moreover, the failure envelop that describe best the experiments data isn’t linear and can hardly be well approximated with only one plane. 

To address the limitations of the simple failure criteria such as that of Mohr-Coulomb, and following the pioneering work of Paul (1968), other investigators, developed a more comprehensive failure criterion that accounts for the three principal stresses, ,  and  (resp. major, intermediate and minor stresses) \cite{Paul1968,Meyer2013}. The bi-linear failure envelop defined by the criterion enables a more accurate prediction of the rock behavior, especially at high stress states. 

This new approach to representation of the rock failure and the corresponding stress state, however, requires model material constants to be evaluated and calibrated using the laboratory tests such those introduced by Labuz et al. 1996 \cite{Labuz1996} and Zeng et al. 2019 \cite{Zeng2019}. Experimentation is a key element in the quest for the most accurate failure criterion, as it accounts for the real response of rocks. In order to be recognized as accurate, the criterion should provide a failure envelop that gives good prediction and approximation of the test results. It is also important that the chosen experiments are diverse and representative of the real state of stress and strain that the rock could undergo. Indeed, a failure criterion well suited for the prediction of a particular test, could lead to a wrong estimation for another test. Therefore, additional test data will help to provide a more accurate prediction of the and evaluation of the model parameters for the failure criterion proposed by Paul (1968) \cite{Paul1968} and forms is the impetus for the present work.

\section{Objective and scope}

The main objective of the work presented in this thesis is to explore the nature of stress states at failure by adding the intermediate stress effect to experiments and investigating the accuracy of three failure criterion. A laboratory testing program was devised to study the mechanical properties of the Dunnville sandstone, to evaluate the existing failure criteria and to update the Paul-Mohr-Coulomb model for this rock. The following define the scope of this thesis:

\begin{enumerate}
    \item Laboratory tests including unconfined compression tests, triaxial compression and extension tests are performed to characterize the rock mechanical properties such friction angle in compression and extension ($\phi_c$ and $\phi_e$, respectively), Young’s modulus ($E$) and Poisson’s ratio ($\nu$).

    \item Most geotechnical engineering problems involves rocks subjected to a plane state of strain. It is particularly the case for tunnels, and other long structures with a constant cross-section and loaded in the plane of the cross-section \cite{Jaeger1979}. A true-triaxial device is used to simulate a plane strain condition by where the minor principal stress ($\sigma_3$) is maintained at a desired target value and the intermediate stress ($\sigma_2$) is increased to develop a condition where $\Delta_{\epsilon_2}=0$ and hence a plane strain condition is simulated.

    \item A true triaxial device is used to develop a stress path to failure where the mean stress is kept constant during the deviatoric loading stage by decreasing the intermediate principal stress ($\sigma_2$) as failure is approached.

    \item The test results from this thesis and those reported in literature used to evaluate the failure criteria available in the published literature.

    \item The test results and a database of available tests will be used to calibrate the model parameters for a Paul-Mohr-Coulomb nonlinear failure criterion which considers the effect of intermediate principal stress.
\end{enumerate}



This project addresses the need of an accurate representation of rock response by bringing new experimental conditions to rock testing, and by investigating a promising failure criterion through the analysis of a large database of experimentation results. 

\section{Thesis organization}

This thesis is organized through the following layout. 

Chapter 2 presents a review of three most widely used failure criteria through their formulation in four different coordinates systems, namely, Mohr-Coulomb, Hoek-Brown and Paul-Mohr-Coulomb. The following representations were chosen:  ($p$-$q$) plane, ($\sigma_1$-$\sigma_3$) plane, as well as the $\pi$-plane and the ($\sigma_1$-$\sigma_2$-$\sigma_3$) space. The two last systems are linked as the $\pi$-plane is a plane perpendicular to the hydrostatic axis ($\sigma_1=\sigma_2=\sigma_3$) . Particular attention is given to the Paul-Mohr-Coulomb criterion, for which the fitting planes construction is detailed. 

Dunnville Sandstone is the chosen test material for the experiment performed in this work. Chapter 3 introduces the rock geologic history as well as its mineralogy, and present procedures and results of initial experiments performed. A detailed description of uniaxial compression tests and conventional triaxial tests is provided. 

Chapter 4 reviews the theoretical background on multiaxial and particularly “true-triaxial” experiments performed during this study. It is followed by a detailed presentation of the Plane-Strain Apparatus and specimen preparation. Finally, the end of the chapter focuses on the procedures and results of plane strain and constant mean stress tests performed. 

In Chapter 5, a complete analysis of the experiment results is presented. The new test data and the existing data from published literature are used to evaluate existing failure criteria, and to calibrate the Paul-Mohr-Coulomb failure criterion.

Finally, Chapter 6 conclude this thesis by summarizing the most important findings of this work.
