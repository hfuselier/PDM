\chapter{Introduction}
\section{Introduction}

The wide variety of rocks that forms the earth crust makes constitutive modeling a challenge. In some sense, each formation can in itself be considered as a different material, with diverse mineralogy, geological history and behavior. However, a common thread is their composite structure formed by solids (minerals), pores, and cracks \cite{Labuz2018}. Years of research on these materials show similarities in results, through the definition of material parameters used to characterize their behavior. Analysis and interpretation of these experimental outcomes are key elements in the understanding of rock response, and enables safer prediction of their behavior, which is used in the design of geotechnical structures. 

Rock response under a certain state of stress is often characterized by its strength. Multiple experiments, such as axisymmetric triaxial and multi axial tests, have been developed with the aim of getting information on the various states of stress that lead to failure. However, site-specific rock engineering properties are often not available during the preliminary design phase of structures constructed on or into the rock mass. Therefore, models are widely used in the initial stage of the design for prediction of the engineering properties of the rock mass. 

Over the last century, many predictive models have been developed for failure. These models describe surfaces in ($\sigma_1$-$\sigma_2$-$\sigma_3$) space that approximate experimental data, where $\sigma_1$, $\sigma_2$ and $\sigma_3$ are principal stresses with no regard to order. The plethora of failure criteria arise form the challenge of finding one that can give the most accurate description of rock behavior. Empirical models are usually developed for a specific rock type or rock formation and therefore, need to be evaluated before they can be applied to a design.

The most well-known and widely used failure criterion is the Mohr-Coulomb (MC) model, which provides a linear relationship between the normal stress and shear stress on the failure plane. The slope of the failure envelope is characterized by the friction angle $\phi$ and the shear-stress intercept, also called cohesion $c$ \cite{Jaeger1979}. MC can also be written in terms of principal stresses $\sigma_I$ and $\sigma_{III}$, respectively the major and minor principal stresses; note that the intermediate stress $\sigma_{II}$ does not appear. Other criterion, such as the Hoek-Brown (HB) model for intact rocks and rock masses, are non-linear in the Mohr and $(\sigma_{III}-\sigma_{I})$ plane \cite{Hoek1980}. HB provides a reasonable estimate of the state of stress at failure, especially for low values confining stress, minor principal stress $\sigma_{III} < C_0/3$ the uniaxial compressive strength. These failure criteria may be written in terms of the major ($\sigma_1$) and minor ($\sigma_3$) principal stresses, without any consideration to effect of intermediate principal stress ($\sigma_2$).

Experiments, have shown that the intermediate principal stress affects the mode of failure and the principal stresses that are developed at failure \cite{Labuz2018,Labuz1996, Zeng2019,Makhnenko2013}. Moreover, the failure envelope that describe best the experimental data is not linear over a large range of mean stress. 

To address the limitations of the $\sigma_I-\sigma_{III}$ failure criteria such as that of Mohr-Coulomb, and following the pioneering work of Paul (1968), other investigators, developed a failure criterion that accounts for the three principal stresses, $\sigma_I$ ,$\sigma_{II}$ and $\sigma_{III}$ \cite{Paul1968,Meyer2013}. The piecewise linear failure surface enables a more accurate prediction of the rock behavior, especially at high mean stress. 

This new approach to represent rock failure and the corresponding stress state requires material constants to be evaluated and calibrated using multi axial strength tests. Experimentation is the key element in the quest evaluating failure criteria. In order to be recognized as accurate, the criterion should provide a failure surface that gives good prediction of the test results. It is also important that the chosen experiments are diverse and representative of the state of stress in the field. Indeed, a failure criterion well suited for the prediction of a particular test could lead to a poor estimate for another test. Therefore, additional test data will help to provide a more accurate evaluation of the model parameters for the failure criterion proposed by Paul (1968) and forms the impetus for the present work \cite{Paul1968}.

\section{Objective and scope}

The main objective of the work presented in this thesis is to explore the nature of stress states at failure as described by the three principal stresses and to investigate the accuracy of three failure criteria. A laboratory testing program was devised to study the mechanical properties of the Dunnville sandstone, to evaluate the existing failure criteria and to calibrate the Paul-Mohr-Coulomb model for this rock. The following define the scope of this thesis:

\begin{enumerate}
    \item Laboratory tests including uniaxial compression, axisymmetric (triaxial) compression and extension tests are performed to characterize the elastic parameters such as Young's modulus ($E$) and Poisson’s ratio ($\nu$), and failure parameters such as friction angle in compression and extension ($\phi_c$ and $\phi_e$, respectively).

    \item Many geo-engineering problems involves rock subjected to a plane state of strain. It is particularly the case for tunnels, and other long structures with a constant cross-section and loaded in the plane of the cross-section \cite{Jaeger1979}. A true-triaxial device is used to simulate a plane strain condition, where the minor principal stress ($\sigma_{III}$) is maintained at a desired target value and the intermediate stress ($\sigma_{II}$) is increased to develop a condition where $\Delta\epsilon_2=0$ is simulated.

    \item A true triaxial device is used to develop a stress path to failure where the mean stress is kept constant during the deviatoric loading stage by decreasing the intermediate principal stress ($\sigma_2$) as failure is approached.

    \item The test results from this thesis and those reported in literature used to evaluate various failure criteria, including the model parameters for the Paul-Mohr-Coulomb (general linear) failure criterion.
\end{enumerate}

\section{Thesis organization}

Chapter \ref{ch2:title} presents a review of three widely used failure criteria namely, Mohr-Coulomb, Hoek-Brown and Paul-Mohr-Coulomb. Chapter \ref{ch3:title} summarizes the geologic history and mineralogy of Dunnville sandstone as well as the results of uniaxial and conventional triaxial (compression and extension) tests on samples of Dunnville sandstone. Chapter \ref{ch4:title} reviews the theoretical background on true triaxial experiments performed in this study, and the results of plane strain and constant mean stress tests are presented. In Chapter \ref{ch5:chapter5}, the new test data and the existing data from published literature are used to (i) evaluate existing failure criteria, and (ii) calibrate the Paul-Mohr-Coulomb failure criterion. Chapter \ref{ch6:chapter6} presents the conclusions and important findings of this thesis. 

The symbols and notations used in this thesis are listed in Appendix \ref{App:A}.