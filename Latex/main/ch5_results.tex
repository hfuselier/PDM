\chapter{Results and Discussion}\label{ch5:chapter5}


\section{Experiments results Database }\label{ch5:database}

The main objective of this study was to propose an evaluation of three failure criteria for a selected rock, Dunnville Sandstone. As explained in the previous chapters, their empirical nature requires a thick database of diverse multi axial experiments for their development. 

Dunnville Sandstone have been used for several works in the past and particularly for multi axial experiments \cite{Labuz2018}\cite{Zeng2019}\cite{Tarokh2016}. The ones performed in the scope of this study (cf. Chapter \ref{ch4:title}) presented the opportunity to enrich the existing tests results database and to evaluate the failure criteria with data representative of Dunnville Sandstone response.

The database presented in Table \ref{tb5:database} was based on the one proposed by Zeng et al. (2019) \cite{Zeng2019}, and extended with the results of the experiments from this study. In this table, each experiment is associated with the following elements: the orientation of the bedding regarding the application of the axial stress, the three principal stresses (i.e. $\sigma_I$,$\sigma_{II}$,$\sigma_{III}$) and the stress invariants (i.e.$p$, $q$, $\theta$).

This database was used for the evaluation of Mohr-Coulomb,Hoek-Brown and Paul-Mohr-Coulomb failure criteria, that will be presented in the following sections.

\begin{table}
    \centering
    \begin{tabular}{cccccccc}
        \hline 
        Test & Bedding & $\sigma_I$ [\si{MPA}] & $\sigma_{II}$ [\si{MPA}] &$\sigma_{III}$ [\si{MPA}] & $p$ [\si{MPA}] & $q$ [\si{MPA}] & $\theta$ [\si{\degree}] \\
        \hline
        \hline
        Published TC-1 & \(\perp\) & 29.7 & 0.0 & 0.0 & 9.9 & 29.7 & 0 \\
        Published TC-2 & \(\perp\) & 39.4 & 2.5 & 2.5 & 14.8 & 36.9 & 0 \\
        Published TC-3 & \(\perp\) & 52.9 & 5.0 & 5.0 & 21.0 & 47.9 & 0 \\
        Published TC-4 & \(\perp\) & 71.5 & 10.0 & 10.0 & 30.5 & 61.5 & 0 \\
        Published TC-5 & \(\perp\) & 98.4 & 20.0 & 20.0 & 46.1 & 78.4 & 0 \\
        Published TC-6 & \(\perp\) & 114.5 & 30.0 & 30.0 & 58.2 & 84.5 & 0 \\
        Published TC-7 & \(\perp\)& 129.4 & 40.0 & 40.0 & 69.8 & 89.4 & 0 \\
        Published TC-8 & \(\perp\) & 142.1 & 50.0 & 50.0 & 80.7 & 92.1 & 0 \\
        Published TC-9 & \(\perp\) & 153.8 & 60.0 & 60.0 & 91.3 & 93.8 & 0 \\
        Published TC-10 & \(\|\) & 24.9 & 0.0 & 0.0 & 8.3 & 24.9 & 0 \\
        Published TC-11 & \(\|\) & 35.2 & 2.5 & 2.5 & 13.4 & 32.7 & 0 \\
        Published TC-12 & \(\|\) & 48.8 & 5.0 & 5.0 & 19.6 & 43.8 & 0 \\
        Published TC-13 & \(\|\) & 68.0 & 10.0 & 10.0 & 29.3 & 58.0 & 0 \\
        Published TC-14 & \(\|\) & 95.9 & 20.0 & 20.0 & 45.3 & 75.9 & 0 \\
        Published TC-15 & \(\|\) & 110.9 & 30.0 & 30.0 & 57.0 & 80.9 & 0 \\
        Published TC-16 & \(\|\) & 125.5 & 40.0 & 40.0 & 68.5 & 85.5 & 0 \\
        Published TC-17 & \(\|\) & 138.1 & 50.0 & 50.0 & 79.4 & 88.1 & 0 \\
        Published TC-18 & \(\|\) & 150.8 & 60.0 & 60.0 & 90.3 & 90.8 & 0 \\
        UCS   & \(\perp\) & 29.8 & 0 & 0   & 27.95 & 51.43 & 0 \\ 
        TC 9  & \(\perp\) & 49.43 & 5  & 5 & 19.81 & 44.43 & 0 \\ 
        TC 0  & \(\perp\) & 61.43 & 10 & 10 & 27.95 & 51.43 & 0 \\ 
        TC 5  & \(\perp\) & 91.08 & 20 & 20 & 44.72 & 71.08 & 0 \\ 
        TC 8  & \(\perp\) & 127.3 & 40 & 40 & 65.73 & 87.30 & 0 \\ 
        TC 10 & \(\perp\) & 151.1 & 60 & 60 & 88.12 & 91.10 & 0 \\ 
        \hline
        \hline
        Published TE-1 & \(\perp\) & 35.0 & 35.0 & 0.8 & 23.6 & 34.2 & 60 \\
        Published TE-2 & \(\perp\) & 40.0 & 40.0 & 1.2 & 27.1 & 38.8 & 60 \\
        Published TE-3 & \(\perp\) & 50.0 & 50.0 & 6.0 & 35.3 & 44.0 & 60 \\
        Published TE-4 & \(\perp\) & 60.0 & 60.0 & 10.1 & 43.4 & 49.9 & 60 \\
        Published TE-5 & \(\perp\) & 69.0 & 69.0 & 11.5 & 49.8 & 57.5 & 60 \\
        Published TE-6 & \(\|\) & 40.0 & 40.0 & 1.8 & 27.3 & 38.2 & 60 \\
        Published TE-7 & \(\|\) & 50.0 & 50.0 & 5.7 & 35.2 & 44.3 & 60 \\
        Published TE-8 & \(\|\) & 60.0 & 60.0 & 8.0 & 42.7 & 52.0 & 60 \\
        TE 3  & \(\perp\) & 35 & 35 & 3.96 & 24.64 & 31.08 & 60 \\ 
        TE 1  & \(\perp\) & 40 & 40 & 4.50 & 27.89 & 36.34 & 60 \\ 
        TE 2  & \(\perp\) & 60 & 60 & 9.68 & 43.01 & 50.98 & 60 \\
        \hline
        \hline
        Published TT-1 & \(\perp\) & 48.3 & 31.6 & 5.0 & 28.3 & 37.8 & 37.5 \\
        Published TT-2 & \(\perp\) & 52.9 & 25.1 & 7.0 & 28.3 & 40.1 & 22.7 \\
        Published TT-3 & \(\perp\) & 63.9 & 12.1 & 9.0 & 28.3 & 53.4 & 2.9 \\
        Published TT-4 & \(\perp\) & 70.6 & 49.4 & 15.0 & 45.0 & 48.7 & 37.8 \\
        Published TT-5 & \(\perp\) & 77.5 & 70.5 & 20.0 & 56.0 & 54.3 & 53.5 \\
        Published TT-6 & \(\perp\) & 83.9 & 62.1 & 22.0 & 56.0 & 54.4 & 39.7 \\
        TT 1 & \(\perp\) & 88.14 & 46.85 & 10 & 48.33 & 55.28 & 28.12 \\
        TT 2 & \(\perp\) & 99.98 & 20 & 20 & 46.65 & 62.28 & 0 \\
        \hline
    \end{tabular}
    \captionsetup{justification=centering}
    \caption{Database of experiments results for Dunnville Sandstone. The "Published" data are from Zeng et al. \cite{Zeng2019}}
    \label{tb5:database}
\end{table}

%%%%%%%%%%%%%%%%%%%%%%%%%%%%%%%%%%%%%%%%%%%%%%%%%%%%%%%%%%%%%%
\section{Evaluation of the failure criteria}\label{ch5:evaluation}

The Mohr-Coulomb, Hoek-Brown and Paul-Mohr-Coulomb failure criteria presented in Chapter \ref{ch2:title} were fitted to the experiment results of Dunnville Sandstone from Table \ref{tb5:database}. A computation program was developed for the fittings using the programming language Python. All the resources needed to access the program files are listed in Appendix REF{APPENDIX B}. 

The three failure criterion fittings are evaluated through their representation in the three coordinates systems presented in Chapter \ref{ch2:title}, and their accuracy in terms of how good they fit the data. In this study, this "accuracy" is chosen to be evaluated by comparing the least mean standard deviation misfits, as proposed by Benz et al. (2008) \cite{Benz2008}. 

The standard deviation $s_{i}$ of one test series $i$ formed by $j$ experiments subject to the same minor stress (i.e. $\sigma_{III}$) is defined by Equation \ref{eq5:standdev}. In this expression, $n$ is the number of experiments in the test series $i$, $\sigma_{I,j}^{\mathrm{test}}$ is the maximum stress at failure for a data point $j$ (obtained from the database) and $\sigma_{I,j}^{\mathrm{calc}}$ is the calculated one using the considered criterion formulation. 

\begin{equation}\label{eq5:standdev}
    s_{i}=\sqrt{\frac{1}{n} \sum_{j}\left(\sigma_{I,j}^{\mathrm{calc}}-\sigma_{I,j}^{\mathrm{test}}\right)^{2}}
\end{equation}

Finally, the mean standard deviation misfit is computed following Equation \ref{eq5:mean_standdev}, where $m$ is the number of test series. The smallest the $\bar{S}$ is , the better is the "accuracy" of the model for the rock compared to other criteria. 

\begin{equation}\label{eq5:mean_standdev}
    \bar{S}=\frac{1}{m} \sum_{i} s_{i}
\end{equation}


%%%%%%%%%%%%%%%%%%%%%%%%%%%%%%%%%%%%%%%%%%%%%%%%%%%%%%%%%%%%%%
\subsection{Mohr-Coulomb failure criterion}

The Mohr-Coulomb failure criterion is formulated in terms two principal stresses (cf. Equations \ref{eq2:MCfinalform} and \ref{eq2:MCcondenseform}) and unique strength parameters (i.e. $\phi$,$c$), therefore, the fitting was done using only axisymmetric triaxial compression tests results (i.e. $\theta = 0^\circ$). 

From this fitting, the coefficient $K_p$ was determined and the other parameters were computed: 
\begin{equation}
    K_p = 2.55 \quad \textrm{and} \quad C_0 = \SI{29.7}{\mega\pascal}
\end{equation}
\begin{equation}
    \phi = \frac{K_p-1}{K_p+1} = \SI{25.9}{\degree}
\end{equation}
\begin{equation}
    c = \frac{C_0(1-sin\phi)}{2cos\phi} = \SI{9.30}{\mega\pascal}
\end{equation}
\begin{equation}
    V_0 = \frac{C_0}{K_p-1} = \SI{19.2}{\mega\pascal}
\end{equation}

Knowing the strength parameters, the Mohr-Coulomb failure surface is plotted in the $(\sigma_3-\sigma_1)$ plane using Equation \ref{eq2:MCfinalform}(cf. Figure \ref{fig5:mc_sig1sig3}).

\begin{figure}[p]
    \centering
    \includegraphics[width=0.7\columnwidth]{ch5/mc_sig1sig3}
    \caption{Mohr-Coulomb criterion failure surface in  $(\sigma_3-\sigma_1)$ plane}
    \label{fig5:mc_sig1sig3}
\end{figure} 

The criterion was also fitted in the $(p-q)$ plane, for which the plot obtained is shown in Figure \ref{fig5:mc_pq}. The coefficients $m_{c,e}$ and $b_{c,e}$ were computed using Equations \ref{eq2:MC_mc_q} to \ref{eq2:MC_be_q}:

\begin{equation}
    m_c = \frac{6 \sin \phi}{3-\sin \phi} = 1.02
\end{equation}
\begin{equation}
    m_e = \frac{6 \sin \phi}{3+\sin \phi} = 0.76
\end{equation}
\begin{equation}
    b_c = \frac{6 c \cos \phi}{3-\sin \phi} = \SI{19.6}{\mega\pascal}
\end{equation}
\begin{equation}
    b_e = \frac{6 c \cos \phi}{3+\sin \phi} = \SI{14.6}{\mega\pascal}
\end{equation}

\begin{figure}[p]
    \centering
    \includegraphics[width=0.7\columnwidth]{ch5/mc_pq}
    \caption{Mohr-Coulomb criterion failure surface in  $(p-q)$ plane}
    \label{fig5:mc_pq}
\end{figure} 

Finally, the Mohr-Coulomb criterion is presented in the $\pi$-plane, obtained following the procedure described in Section \ref{ch2:MC_pi}. Figure \ref{fig5:mc_pi_plane} shows Mohr-Coulomb failure criterion in the pi-plane at different values of the mean stress $p$.

\begin{figure}[tb]
    \centering
    \includegraphics[width=0.5\columnwidth]{ch5/mc_pi_plane1}
    \includegraphics[width=0.8\columnwidth]{ch5/mc_pi_plane2}
    \caption{Mohr-Coulomb criterion failure surface in  $\pi$-plane}
    \label{fig5:mc_pi_plane}
\end{figure} 

The mean standard deviation misfit obtained with the Mohr-Coulomb failure criterion is $14.0$ [\si{\mega\pascal}]. 

%%%%%%%%%%%%%%%%%%%%%%%%%%%%%%%%%%%%%%%%%%%%%%%%%%%%%%%%%%%%%%
\subsection{Hoek-Brown failure criterion}

The Hoek-Brown failure criterion is also formulated in terms two principal stresses (cf. Equations \ref{eq2:HB-crit}) and unique strength parameters (i.e. $m$, $C_0$), therefore, the fitting was done using only axisymmetric triaxial compression tests results (i.e. $\theta = 0^\circ$). 

From this fitting, the strength parameter $m$ was determined and $V_0$ was computed: 

\begin{equation}
    m = 5.96 \quad \textrm{and} \quad C_0 = \SI{29.7}{\mega\pascal}
\end{equation}
\begin{equation}
    V_0 = \frac{C_0}{m} = \SI{4.98}{\mega\pascal}
\end{equation}

Knowing the strength parameters, the Hoek-Brown failure surface is plotted in the $(\sigma_3-\sigma_1)$ plane using Equations \ref{eq2:HBsig1_CTC} for the compression line and \ref{eq2:HBsig1_CTE} for extension (cf. Figure \ref{fig5:mc_sig1sig3}).

\begin{figure}[p]
    \centering
    \includegraphics[width=0.7\columnwidth]{ch5/hb_sig1sig3}
    \caption{Hoek-Brown criterion failure surface in  $(\sigma_3-\sigma_1)$ plane}
    \label{fig5:hb_sig1sig3}
\end{figure} 

In the $(p-q)$ plane, the Hoek-Brown failure criterion is plotted using Equations \ref{eq2:HB-q-CTC} for compression and \ref{eq2:HB-q-CTE} for extension, and showed in Figure \ref{fig5:hb_pq}. These surfaces are expressed in terms of $m$ and $C_0$ previously defined. 

\begin{figure}[p]
    \centering
    \includegraphics[width=0.7\columnwidth]{ch5/hb_pq}
    \caption{Hoek-Brown criterion failure surface in  $(p-q)$ plane}
    \label{fig5:hb_pq}
\end{figure} 

Finally, the Hoek-Brown criterion is presented in the $\pi$-plane, obtained following the procedure described in Section \ref{ch2:HB_pi}. Figure \ref{fig5:hb_pi_plane} shows Hoek-Brown failure criterion in the pi-plane at different values of the mean stress $p$.

\begin{figure}[tb]
    \centering
    \includegraphics[width=0.5\columnwidth]{ch5/hb_pi_plane1}
    \includegraphics[width=0.8\columnwidth]{ch5/hb_pi_plane2}
    \caption{Hoek-Brown criterion failure surface in  $\pi$-plane}
    \label{fig5:hb_pi_plane}
\end{figure} 

The mean standard deviation misfit obtained with the Hoek-Brown failure criterion is $9.37$ [\si{\mega\pascal}]. 

%%%%%%%%%%%%%%%%%%%%%%%%%%%%%%%%%%%%%%%%%%%%%%%%%%%%%%%%%%%%%%
\subsection{Paul-Mohr-Coulomb failure criterion}\label{ch5:3p_pmc}

The fitting presented in this section is referred as the "3 parameters Paul-Mohr-Coulomb criterion", as it accounts only 
Contrary to the previous criteria, the Paul-Mohr-Coulomb failure criterion is formulated in terms the three principal stresses (cf. Equations \ref{eq2:PMC} and \ref{eq2:PMC_dev}) and non-unique strength parameters (i.e. $\phi_{c,e}$, $c_{c,e}$, $V_0$), therefore, the fitting was done using all tests results from the database (i.e. $\theta = 0^\circ$). 

From the least-square solution fitting describe in Chapter 2 (cf. Section \ref{ch2:pmcfit}, Equations \ref{eq2:PMCfinalform} and \ref{eq2:pmc_be}), the following solution could be obtained:
\begin{equation}
    x_1 = \frac{b_c}{V_0} = 0.81
\end{equation}
\begin{equation}
    x_2 = k = -0.91
\end{equation}
\begin{equation}\label{eq5:pmc_bc}
    x_3 = b_c = \SI{28.77}{\mega\pascal}
\end{equation}

Following Equations \ref{eq2:pmc_vo} to \ref{eq2:pmc_c}, the strength parameters for Paul-Mohr-Coulomb failure criteria could be computed:

\begin{equation}
    V_0 = \frac{0.81}{b_c} = \SI{35.62}{\mega\pascal}
\end{equation}
\begin{equation}\label{eq5:pmc_be}
    b_e = \frac{2b_c}{(1-\sqrt{3}k)} = \SI{22.31}{\mega\pascal}
\end{equation}
\begin{equation}
    \phi_c = arcsin\left(\frac{3b_c}{6V_0+b_c}\right) = \SI{20.85}{\degree}
\end{equation}
\begin{equation}
    \phi_e = arcsin\left(\frac{3b_e}{6V_0-b_c}\right) = \SI{20.85}{\degree}
\end{equation}
\begin{equation}
    c_{c}=\frac{b_{c}\left(3-\sin \phi_{c}\right)}{6 \cos \phi_{c}} = \SI{13.57}{\mega\pascal}
\end{equation}
\begin{equation}
    c_{e}=\frac{b_{e}\left(3+\sin \phi_{e}\right)}{6 \cos \phi_{e}} = \SI{10.52}{\mega\pascal}
\end{equation}

Knowing the strength parameters, the Paul-Mohr-Coulomb failure surface could be plotted in the $(\sigma_3-\sigma_1)$ plane using Equations \ref{eq2:PMC_sig1sig3} to \ref{eq2:PMC_sig1sig3_Cce}. The graph obtained, using the coefficients computed in Equations \ref{eq5:pmc_Mc} to \ref{eq5:pmc_Ce}, is presented in Figure \ref{fig5:pmc_sig1sig3}.

\begin{equation}\label{eq5:pmc_Mc}
    M_c = \frac{1+\sin \phi_c}{1-\sin \phi_c} = 2.11
\end{equation}
\begin{equation}
    M_e = \frac{1+\sin \phi_e}{1-\sin \phi_e} = 2.08
\end{equation}
\begin{equation}
    C_c = \frac{2c_c\cos \phi_c}{1-\sin \phi_c} = \SI{39.38}{\mega\pascal}
\end{equation}
\begin{equation}\label{eq5:pmc_Ce}
    C_e = \frac{2c_e\cos \phi_e}{1-\sin \phi_e} = \SI{30.31}{\mega\pascal}
\end{equation}

\begin{figure}[p]
    \centering
    \includegraphics[width=0.7\columnwidth]{ch5/pmc3p_sig1sig3}
    \caption{Paul-Mohr-Coulomb criterion failure surface in  $(\sigma_3-\sigma_1)$ plane}
    \label{fig5:pmc_sig1sig3}
\end{figure}

In the $(p-q)$ plane, the Paul-Mohr-Coulomb failure criterion is plotted using Equations \ref{eq2:PMC_pq} to \ref{eq2:pmc_b_pq}, and the graph obtained in presented in Figure \ref{fig5:pmc_pq}. These surfaces are expressed in terms of $b_{c,e}$,defined by Equations \ref{eq5:pmc_bc} and \ref{eq5:pmc_be}, and $m_{c,e}$ computes as follow: 

\begin{equation}\label{eq5:pmc_mc}
    m_c=\frac{6 \sin \phi_{c}}{3-\sin \phi_{c}} = 0.81
\end{equation}

\begin{figure}[p]
    \centering
    \includegraphics[width=0.7\columnwidth]{ch5/pmc3p_pq}
    \caption{Paul-Mohr-Coulomb criterion failure surface in $(p-q)$ plane}
    \label{fig5:pmc_pq}
\end{figure} 

Finally, the Paul-Mohr-Coulomb criterion is presented in the $\pi$-plane, obtained following the procedure described in Section \ref{ch2:PMC}. Figure \ref{fig5:pmc_pi_plane} shows the failure criterion in the $\pi$-plane at different mean stresses $p$, corresponding to true-triaxial experiments mean stresses at failure (i.e. data points where $0^\circ < \theta < 60^\circ$ in Table \ref{tb5:database}).

\begin{figure}[tb]
    \centering
    \includegraphics[width=0.5\columnwidth]{ch5/pmc3p_pi_pts1}
    \includegraphics[width=0.8\columnwidth]{ch5/pmc3p_pi_pts2}
    \caption{Paul-Mohr-Coulomb criterion failure surface in  $\pi$-plane}
    \label{fig5:pmc_pi_plane}
\end{figure} 

The mean standard deviation misfit obtained with the Paul-Mohr-Coulomb failure criterion is $13.2$ [\si{\mega\pascal}]. 

%%%%%%%%%%%%%%%%%%%%%%%%%%%%%%%%%%%%%%%%%%%%%%%%%%%%%%%%%%%%%%
\subsection{Comparison of the failure criteria}

The mean standard deviation misfits obtained for the three failure criteria show that Hoek-Brown provide a better approximation of the data points (cf. Table \ref{tb5:stand_dev}). However, it should be kept in mind that this criterion was fitted only for data points related to axisymmetric compression experiments. Therefore, its prediction of true triaxial experiments is less accurate than the one provided by the Paul-Mohr-Coulomb criterion. This can be easily noticed with the observation of both predictions in the $\pi$-plane (cf. Figures \ref{fig5:hb_pi_plane} and \ref{fig5:pmc_pi_plane}). 

\begin{table}
    \centering
    \begin{tabular}{ccc}
        \hline 
        Criterion & Mean standard deviation misfit $\bar{S}$ \\
        \hline
        \hline
        Mohr-Coulomb & 14.0 \\
        Hoek-Brown & 9.37 \\
        Paul-Mohr-Coulomb & 13.2 \\
        \hline
    \end{tabular}
    \captionsetup{justification=centering}
    \caption{Summary of the mean standard deviation misfits obtained for the three failure criteria evaluated}
    \label{tb5:stand_dev}
\end{table}

%%%%%%%%%%%%%%%%%%%%%%%%%%%%%%%%%%%%%%%%%%%%%%%%%%%%%%%%%%%%%%%%%%%%%%%%%%%%%%%%%%%%%%%%%%%%%%%%%%%%%%%%%%%%%%%%%%%%%%%%%%%%%%%%%%%%%%%%%%%
\section{Bi-linear Paul-Mohr-Coulomb failure criterion}\label{ch5:PMC}

Published data from multi axial experiments on multiple rocks showed that the failure envelop that describe them best is not linear over a large range of mean stress. However, popular failure theories as Mohr-Coulomb or Hoek-Brown, are either linear or do not provide an accurate prediction for all mean stresses. In this study, Paul-Mohr-Coulomb failure criteria is chosen to respond to this issue by approximating the nonlinear failure surface in a piecewise linear manner, resulting in a failure surface defined by six parameters.

\subsection{Paul-Mohr-Coulomb with six parameters}

The theoretical background on the six parameters Paul-Mohr-Coulom criterion presented in this section is based on the work of Labuz et al. (2018) \cite{Labuz2018}.

The Paul-Mohr-Coulomb failure criterion presented in the Section \ref{ch5:3p_pmc} is referred to as Paul-Mohr-Coulomb with three parameters, which failure surface is a plane defined by the general equation of the criterion (cf. Equation \ref{eq2:PMC_dev}), using three strength parameters : $V_0$, $\phi_c$ and $\phi_e$. The Paul-Mohr-Coulomb failure surface defined in a piecewise manner is, therefore, made of a minimum of two plane each expressed using three strength parameters, leading to the six parameters criterion.

The three parameters criterion definition in terms of the three principal stresses describes a regular 6-sided pyramid in the principal stresses three-dimensional space (cf. Section \ref{ch2:background}). Therefore, by adding a plane to the failure surface, the six parameters criterion describes two irregular 6-sided pyramids. Each plane is then defined by the parameters presented in Table \ref{tb5:pmc6p_planeparam}, where $P2$ indicates the plane that approximate data points at low mean stress and $P1$ the ones at higher mean stress. Table \ref{tb5:pmc6p_pyramids} present four types of Paul-Mohr-Coulomb failure surfaces which can be defined according to the values of the parameters. For all types, the following conditions apply: 
\begin{equation}
    V_0^{(1)} > V_0^{(2)} \quad \textrm{and} \quad \SI{0}{\degree} \leq \phi_{c,e}^{(i)} \leq \SI{60}{\degree}
\end{equation}

\begin{table}
    \centering 
    \begin{tabular}{ccc}
        \hline 
        Plane & $P1$ & $P2$  \\
        \hline
        \hline
        Friction angle in compression & $\phi_{c}^{(1)}$ & $\phi_{c}^{(2)}$ \\
        \\
        Friction angle in extension & $\phi_{e}^{(1)}$ & $\phi_{e}^{(2)}$ \\ 
        \\
        Theoretical uniaxial tensile strength & $V_0^{(1)}$ & $V_0^{(2)}$ \\
        \hline
    \end{tabular}
    \captionsetup{justification=centering}
    \caption{Parameters of the planes defining the failure surface of Paul-Mohr-Coulomb criterion}
    \label{tb5:pmc6p_planeparam}
\end{table}

\begin{table}
    \centering 
    \begin{tabular}{cc}
        \hline 
        Type of failure surface & Parameters conditions   \\
        \hline
        \hline
        (i) 6-sided & $V_0^{(1)} = V_0^{(2)}$ \\
        \\
        (ii) 6-12-6 sided & $\phi_{c}^{(1)} < \phi_{c}^{(2)}$, $\phi_{e}^{(1)}$ < $\phi_{e}^{(2)}$, $p_c \neq p_e$\\ 
        \\
        (iii) 6-12 sided & ($\phi_{c}^{(1)} < \phi_{c}^{(2)}$, $\phi_{e}^{(1)} \geq \phi_{e}^{(2)}$) or ($\phi_{c}^{(1)} \geq \phi_{c}^{(2)}$, $\phi_{e}^{(1)} < \phi_{e}^{(2)}$)\\
        \\
        (iv) 6-12-6 sided & $\phi_{c}^{(1)} < \phi_{c}^{(2)}$, $\phi_{e}^{(1)}$ < $\phi_{e}^{(2)}$, $p_c = p_e$\\ 
        \hline
    \end{tabular}
    \captionsetup{justification=centering}
    \caption{Parameters of the planes defining the failure surface of Paul-Mohr-Coulomb criterion}
    \label{tb5:pmc6p_pyramids}
\end{table}

The complete graphical representation of the Paul-Mohr-Coulomb failure surface is composed of the $(p-q)$ plane, the $(\sigma_2-\sigma_1)$ plane, the $\pi$-plane and the principal stresses three-dimensional space. The transition between $P2$ and $P1$ is well represented in the $(p-q)$ plane, where they intersect on the compression side (i.e. $q > 0$) at the mean stress value $p_c$ and on the extension side (i.e. $q < 0$) at $p_e$. In the case of the failure surface type (ii), these transitions points have different values leading to a 12 sided transition zone on the pyramid for mean stress values $p \in \left[p_c;p_e\right]$. Sketches of the 6-12-6 sided failure surface in different planes and in the three-dimensional space are presented in Figure \ref{fig5:6pmc_typeii}, where the $(\sqrt{3}p-\sigma^*)$ plane is equivalent to the $(p-q)$ plane. More schematic representations and details on the four failure surfaces types are provided in APPENDIX B REF{APPENDIX B}.

\begin{figure}
    \centering
    \includegraphics[width=\columnwidth]{ch5/pmc6_typeii}
    \caption{Paul-Mohr-Coulomb 6-12-6 sided failure surface graphical representations}
    \label{fig5:6pmc_typeii}
\end{figure}

The addition of the intermediate stress in the Paul-Mohr-Coulomb general equation (cf. Equation \ref{eq2:PMC}) makes relevant the representation of the criterion in the $(\sigma_2-\sigma_1)$ plane. Indeed, this plane presents the advantage to gather the data points for axisymmetric experiments, shown on compression and extension lines, as well as true-triaxial data in the same plot (cf. Figure \ref{fig5:6pmc_sig2sig1}). The Paul-Mohr-Coulomb failure surfaces are plotted for a chosen value of $\sigma_3$, using the following equation, based the rearrangement of Equation \ref{eq2:PMC}:

\begin{equation}\label{eq5:pmc_sig2sig1}
    \sigma_1 = \frac{1}{A}\left(1-B\sigma_{2}-C\sigma_{3}\right)
\end{equation}
\begin{figure}
    \centering
    \includegraphics[width=0.6\columnwidth]{ch5/pmc6_sig2sig1_ske}
    \caption{Paul-Mohr-Coulomb 6-12-6 sided failure surface in $(\sigma_2-\sigma_1)$ plane}
    \label{fig5:6pmc_sig2sig1}
\end{figure}

%%%%%%%%%%%%%%%%%%%%%%%%%%%%%%%%%%%%%%%%%%%%%%%%%%%%%%%%%%%%%%%%%%%%%%%%%%%%%%%%%%
\subsection{Bi-linear fitting program}

The fitting of a six-parameters Paul-Mohr-Coulomb failure surface requires to create two data sets from the initial database of experiments results, one for each plane. Once defined, these data sets are used in for the fitting of each plane, following the procedure presented in Section \ref{ch2:pmcfit}. This repartition of the data points into different planes is a challenging step of the failure criterion fitting, as it should give the optimal solution for the database considered. 
****** Smthg about the fact that we want to develop a solution that works for all databases (i.e. for any rocks) *********

It can be summarize with the following question: 

How to automatize the repartition of data point into $P1$ or $P2$ ? 

In the following section, the analytical theory used to create the program will be describe.
The solution developed was implemented in a program using the programming language Python. 

\subsubsection{$Brute\_force.py$}



\subsection{Dunnville Sandstone}


%%%%%%%%%%%%%%%%%%%%%%%%%%%%%%%%%%%%%%%%%%%%%%%%%%%%%%%%%%%%%%%%%%%%%%%%%%%%%%%%%%%%%%%%%%%%%%%%%%%%%%%%%%%%%%%%%%%%%%%%%%%%%%%%%%%%%%%%%%%
\section{Discussion}